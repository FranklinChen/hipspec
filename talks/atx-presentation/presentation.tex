%!TEX TS-program = xelatex
%!TEX encoding = UTF-8 Unicode

\documentclass[serif,professionalfont]{beamer}
\def\maindocument{} % To tell tikz images that they are not stand alone

\usepackage{tikz}
\usetikzlibrary{shapes,arrows,calc}

% \usepackage[utf8]{inputenc}
% \usepackage[T1]{fontenc}
%\usepackage{fixltx2e}
%\usepackage{graphicx}
%\usepackage{longtable}
%\usepackage{float}
%\usepackage{wrapfig}
%\usepackage{soul}
%
%\usepackage{textcomp}
%\usepackage{marvosym}
%\usepackage{wasysym}
%\usepackage{latexsym}
\usepackage{amssymb}

\usepackage{hyperref}

\usepackage{mathpartir}
\usepackage{color}
%\tolerance=1000
\usepackage{inconsolata}
\usepackage{amsmath}
\usepackage{array}
\providecommand{\alert}[1]{\textbf{#1}}

\definecolor{Purpleee}{RGB}{140,20,140}
\definecolor{PurpleL}{RGB}{255,170,255}
\definecolor{MyGreen}{RGB}{5,95,5}
\setbeamercolor{title}{fg=Purpleee}
\setbeamercolor{frametitle}{fg=Purpleee}
\setbeamercolor{structure}{fg=Purpleee}


\TeXXeTstate=1
\usepackage{mathspec,xltxtra,xunicode}
% \setsansfont{Gill Sans}

\setmainfont[Scale=1]{Gill Sans}
%\setmonofont[Scale=0.8]{Monaco}
\setmonofont{Inconsolata}

\setmathsfont(Digits,Latin,Greek){Gill Sans}

\usefonttheme[onlymath]{serif}

%\usepackage{fontspec}
%\defaultfontfeatures{Mapping=tex-text}
%\setsansfont[Ligatures={Common}]{Futura}


\usepackage{listings}

\lstnewenvironment{codex}[1][]%
  {
   \noindent
   \minipage{\linewidth}
   \vspace{0.2\baselineskip}
%   \vspace{-0.4\baselineskip}
   \lstset{basicstyle=\ttfamily,
%           frame=single,
           language=Haskell,
           keywordstyle=\color{black},
           #1}}
  {%\vspace{-0.8\baselineskip}
   \endminipage}

\title{ {\Huge HipSpec} }
\subtitle{ {\LARGE Automating Inductive Proofs \\ of Program Properties} }
\institute{ Chalmers University of Technology | University of Gothenburg }

\author{ {\Large Dan Ros\'en}
     \vspace{\baselineskip} \\
     Koen Claessen, Moa Johansson, Nicholas Smallbone }
\date{July 1, 2012}

\newcommand\fa[1]{ \forall \, #1 . \,}
\newcommand\faa[2]{ \forall \, #1 , #2 . \,}
\newcommand\faaa[3]{ \forall \, #1 , #2 , #3 . \,}
\newcommand\faaaa[4]{ \forall \, #1 , #2 , #3 , #4 . \,}
\newcommand\up[0]{\vspace{-\baselineskip}}
\newcommand\dn[0]{\vspace{\baselineskip}}
\newcommand\hs[1]{\texttt{#1}}

\newcommand\x[0]{\hs{x}}
\newcommand\xs[0]{\hs{xs}}
\newcommand\ys[0]{\hs{ys}}
\newcommand\xxs[0]{\hs{x:xs}}
\newcommand\nil[0]{\hs{[]}}
\newcommand\p[1]{\textsf{P}(#1)}

\newcommand{\highlight}[1]{\colorbox{PurpleL}{\ensuremath{#1}}}

\begin{document}

\maketitle

\begin{frame}[fragile]
  \frametitle{What is HipSpec?}
\tikzstyle{block} = [rectangle, draw=Purpleee, thick, text width=4.75em, text centered]


\makebox[\textwidth][c]{
\begin{tikzpicture}

  \node at (0,0) [block,text width=120] (src) {
\\
\textbf{Haskell source}
{\small
\begin{align*}
&\hs{rev [] = []} \\
&\hs{rev (x:xs)} \\
&\quad\hs{= rev xs ++ [x]} \\ \\
&\hs{prop\_rev xs} \\
&\quad\hs{= rev (rev xs) =:= xs}
\end{align*}
}
};
\pause
  \node at (5,2) [block,text width=120] (hip) {
\textbf{Hip}
\\
\emph{Haskell Inductive Prover}
\begin{itemize}
\item FOL translation
\item Apply induction
\item Success\pause, or stuck!
\end{itemize}
};
\invisible<1-3>{
  \node at (5,-2)  [block,text width=120] (qs) {
\textbf{QuickSpec}
\\
Eq-theory from testing:
\vspace{-0.5\baselineskip}
{\small
\begin{align*}
&\hs{rev (xs ++ ys)} \\
&\quad\hs{= rev ys ++ rev xs} \\
&\hs{xs ++ [] = []} \\
&\hs{xs ++ (ys ++ zs) =} \\
&\quad\hs{(xs ++ ys) ++ zs}
\end{align*}
}};

  \onslide<5>{
    \node at (8.5,-2) [block] (lemmas)
       {\textbf{HipSpec} \\ \emph{Use these as lemmas!!}};
  }

\end{tikzpicture}
}
}

\end{frame}


\begin{frame}[fragile]
  \frametitle{Example functional program and property}
  %\begin{center}

    \begin{verbatim}
        rev (x:xs) = rev xs ++ [x]
        rev []     = []

        qrev (x:xs) ys = qrev xs (x:ys)
        qrev []     ys = ys
    \end{verbatim}

    \begin{equation*}
    \text{Goal:} \quad \fa{\xs} \hs{rev xs} = \hs{qrev xs []}
    \end{equation*}

%    \pause
%
%       \begin{align*}
%         \text{lhs:} \quad & \hs{rev xs} = ... \\
%         \text{rhs:} \quad & \hs{qrev xs []} = ...
%       \end{align*}
%
%    \pause
%
%    \begin{center}
%      {\color{red} Stuck!}
%    \end{center}

\end{frame}

%   \begin{align*}
%     & \text{to show: } && \hs{rev (x:xs)} = \hs{qrev (x:xs) []} \\
%     \\
%     & \text{lhs:}      && \hs{rev (x:xs)} = \hs{rev xs ++ [x]} \\
%     &                  && = \hs{qrev xs [] ++ [x]} \\
%     & \text{rhs: }     && \hs{qrev (x:xs) []} = \hs{qrev xs [x]}
%   \end{align*}

%\begin{frame}[fragile]
%  \frametitle{Case split}
%
%  In a well-typed language, any list such as $\xs$ is either nil or cons!
%      \begin{mathpar}
%        \inferrule*
%           {
%             \p{\nil}
%             \\
%             \faa{\x}{\xs} \p{\xxs}
%           }
%           { \fa{\xs} \p{\xs} }
%      \end{mathpar}
%
%  \pause
%
%  \begin{equation*}
%   \p{\xs} \Leftrightarrow \hs{rev xs} = \hs{qrev xs []}
%  \end{equation*}
%
%  \invisible<1-2>{
%  \only<1-3>{
%
%     Nil case:
%
%     \begin{align*}
%       \text{lhs:} \quad & \hs{rev []} = \nil  \\
%       \text{rhs:} \quad & \hs{qrev [] []} = \nil
%     \end{align*}
%
%     \pause
%     \begin{center}{\color{MyGreen} Hooray!}\end{center}
%  }
%  \only<4-5>{
%     Cons case:
%
%     \begin{align*}
%       \text{lhs:} \quad & \hs{rev (x:xs)} = \hs{rev xs ++ [x]} \\
%       \text{rhs:} \quad & \hs{qrev (x:xs) []} = \hs{qrev xs [x]}
%     \end{align*}
%
%     \onslide<5>{\begin{center}{\color{red} Stuck!!!}\end{center}}
%  }
%  }
%\end{frame}

\begin{frame}[fragile]
  \frametitle{Structural induction}

  \begin{mathpar}
    \inferrule*
       {
         \p{\nil}
         \\
         \faa{\x}{\xs} \p{\xs} \implies \p{\xxs}
       }
       { \fa{\xs} \p{\xs} }
  \end{mathpar}

  \pause

  \begin{align*}
    \text{lhs:} \quad & \hs{rev (x:xs)} = \hs{rev xs ++ [x]} \onslide<4->{\highlight{ = \hs{qrev xs [] ++ [x]}}} \\
    \text{rhs:} \quad & \hs{qrev (x:xs) []} = \hs{qrev xs [x]}
  \end{align*}

  \pause

  \begin{equation*}
  \textsf{hypothesis:} \quad \hs{rev xs} = \hs{qrev xs []}
  \end{equation*}

  \pause
  \pause
  \begin{center}{\color{red} \Large Stuck!!!}\end{center}

\end{frame}
\begin{frame}[fragile]
  \frametitle{How do we proceed?}

    \begin{verbatim}
        rev (x:xs) = rev xs ++ [x]
        rev []     = []

        qrev (x:xs) ys = qrev xs (x:ys)
        qrev []     ys = ys
    \end{verbatim}

  \up\up

  \begin{align*}
    \text{lhs:} \quad & \hs{rev (x:xs)} = \hs{rev xs ++ [x]}  = \hs{qrev xs [] ++ [x]} \\
    \text{rhs:} \quad & \hs{qrev (x:xs) []} = \hs{qrev xs [x]}
  \end{align*}

  \pause

  What about...
  \begin{equation*}
    \textsf{New Goal:} \quad \faa{\xs}{\highlight{\ys}} \hs{rev xs}\highlight{\hs{ ++ ys}} = \hs{qrev xs }\highlight{\ys}
  \end{equation*}

  \pause

  Then with $\ys = \nil$, we get
  \begin{equation*}
    \onslide<4>{\hs{rev xs} = }\hs{rev xs ++ []} = \hs{qrev xs []}
  \end{equation*}

\end{frame}


\begin{frame}[fragile]
  \frametitle{The crucial lemma}

  \begin{equation*}
  \textsf{New Goal:} \quad \faa{\xs}{\ys} \hs{rev xs ++ ys} = \hs{qrev xs ys}
  \end{equation*}


  \vspace{\baselineskip}

  Induction on $\xs$

  \dn \pause

  Base, to show: $\fa{\ys} \hs{rev [] ++ ys} = \hs{qrev xs []}$

  \begin{align*}
    \text{lhs:} \quad & \hs{rev [] ++ ys} = \ys  \\
    \text{rhs:} \quad & \hs{qrev [] ys} = \ys
  \end{align*}

   \pause

  {\begin{center}{\color{MyGreen} Hooray!}\end{center}}
\end{frame}

\begin{frame}[fragile]
  \frametitle{The crucial lemma}
  Step, to show: $\fa{\ys} \hs{rev (x:xs) ++ ys} = \hs{qrev (x:xs) ys}$

  \dn

  Hypothesis: $\fa{\ys} \hs{rev xs ++ ys} = \hs{qrev xs ys}$

  \pause

  \begin{align*}
  \text{lhs } = \;\; & \hs{rev (x:xs) ++ ys}      && =\{\text{definition of \hs{rev}}\} \\
                     & \hs{(rev xs ++ [x]) ++ ys} && =\{\text{associativity of \hs{++}}\} \\
                     & \hs{rev xs ++ ([x] ++ ys)} && =\{\text{definition of \hs{++}}\} \\
                     & \hs{rev xs ++ (x:ys)}      && =\{\text{induction hypothesis on $\hs{(x:ys)}$}\} \\
                     & \hs{qrev xs (x:ys)}        && =\{\text{definition of \hs{qrev}}\} \\
                     & \hs{qrev (x:xs) ys}        && = \text{rhs}
  \end{align*}

  \pause
  {\begin{center}{\color{MyGreen} \Large HOORAY!}\end{center}}

\end{frame}

\begin{frame}[fragile]
  \frametitle{Success}

     We managed to prove

     \begin{equation*}
     \fa{\xs} \hs{rev xs} = \hs{qrev xs []}
     \end{equation*}

     Using:

     \begin{itemize}
       \item $\faa{\xs}{\ys} \hs{rev xs ++ ys} = \hs{qrev xs ys}$
       \item Induction
         \pause
       \item But we also needed associativity of \hs{++} and $\fa{\xs} \hs{xs ++ []} = \xs$,
             which need induction to be proved
     \end{itemize}
\end{frame}

\begin{frame}
  \frametitle{Setting}

  Prove properties of functional programs using rewriting and
  induction.

  \dn

  {\color{Purpleee} Problem:} Some of these properties require lemmas, that

  \begin{itemize}
    \item Needs to be conjectured,
    \item Requires induction to be proved, and
    \item Might require lemmas themselves
  \end{itemize}

\end{frame}

\begin{frame}
  \frametitle{Enter HipSpec}

  Solves this problems by:

  \begin{itemize}
    \item Generates an equational theory by counter-example testing,
    \item Try to prove this theory by applying induction
    \item Then, try to prove the user-stated properties
    \item Proof search with first-order theorem provers
  \end{itemize}
\end{frame}

\begin{frame}
  \frametitle{Overview of HipSpec}
  \newcommand{\h}[2]{\action<#1->{#2}}

% Define block styles
\tikzstyle{bigblock}   = [rectangle, draw=Purpleee, thick,
                          text width=4.75em, text centered,
                          minimum height=10em,
                          rounded corners=.8ex]

\tikzstyle{smallblock} = [rectangle,
                          text width=6em, text centered,
                          minimum height=3em,thick,
                          draw=Purpleee,rounded corners=.8ex]

\tikzstyle{arr} = [->,>=stealth',semithick]

\makebox[\textwidth][c]{
\begin{tikzpicture}

    % Place nodes
    \node at (-0.75,0)     [bigblock] (src) {Program};
    \pause

    \node at (4,1.5)  [smallblock] (conj) {Conjectures};
    \draw [arr] (src)  to [bend left=15] node [above]
       {
         QuickSpec
       } (conj);

    \pause
    \node at (4,-1.5) [smallblock] (thy)  {First-Order Theory};
    \draw [arr] (src)  to [bend left=-15] node [below,text centered,text width=5em]
       {
         Translation (Hip)
       } (thy);
    \pause
    \node at (8.5,0)     [smallblock,text width=8em] (atp)  {Theorem Prover \\ {\small (E, Vampire, Z3...)}};

    \node at (4.5,0) (text) {Induction (Hip)};


    \draw [arr] (conj) to [bend right=17]  (atp);
    \draw [arr] (thy)  to [bend right=-17] (atp);

    \pause
    \draw [arr] (atp)  to [bend left=15]
        node [right,near start] {\texttt{ Theorem}}
        node [below right,very near end] {Extend theory} (thy);

    \pause
    \draw [arr] (atp)  to [bend left=-15]
        node [right,near start] {\texttt{ Timeout}}
        node [above right,very near end] {Open conjecture} (conj);

\end{tikzpicture}
}



\end{frame}

\begin{frame}
  \frametitle{Overview of HipSpec}
  \def\hlqs{} % Highlight QuickSpec
  \newcommand{\h}[2]{\action<#1->{#2}}

% Define block styles
\tikzstyle{bigblock}   = [rectangle, draw=Purpleee, thick,
                          text width=4.75em, text centered,
                          minimum height=10em,
                          rounded corners=.8ex]

\tikzstyle{smallblock} = [rectangle,
                          text width=6em, text centered,
                          minimum height=3em,thick,
                          draw=Purpleee,rounded corners=.8ex]

\tikzstyle{arr} = [->,>=stealth',semithick]

\makebox[\textwidth][c]{
\begin{tikzpicture}

    % Place nodes
    \node at (-0.75,0)     [bigblock] (src) {Program};
    \pause

    \node at (4,1.5)  [smallblock] (conj) {Conjectures};
    \draw [arr] (src)  to [bend left=15] node [above]
       {
         QuickSpec
       } (conj);

    \pause
    \node at (4,-1.5) [smallblock] (thy)  {First-Order Theory};
    \draw [arr] (src)  to [bend left=-15] node [below,text centered,text width=5em]
       {
         Translation (Hip)
       } (thy);
    \pause
    \node at (8.5,0)     [smallblock,text width=8em] (atp)  {Theorem Prover \\ {\small (E, Vampire, Z3...)}};

    \node at (4.5,0) (text) {Induction (Hip)};


    \draw [arr] (conj) to [bend right=17]  (atp);
    \draw [arr] (thy)  to [bend right=-17] (atp);

    \pause
    \draw [arr] (atp)  to [bend left=15]
        node [right,near start] {\texttt{ Theorem}}
        node [below right,very near end] {Extend theory} (thy);

    \pause
    \draw [arr] (atp)  to [bend left=-15]
        node [right,near start] {\texttt{ Timeout}}
        node [above right,very near end] {Open conjecture} (conj);

\end{tikzpicture}
}



  \undef\hlqs
\end{frame}

\newcommand\vb[1]{\mbox{{\tt #1}}}

\tikzstyle{eqblock} = [rectangle, draw=Purpleee, thick, text width=4.75em]

\begin{frame}[fragile]
\frametitle{Equivalence classes partitioning}

Generates a bunch of terms:

\dn


\begin{tabular}{>{\footnotesize}l >{\footnotesize}l >{\footnotesize}l >{\footnotesize}l}
\vb{[]}&
\vb{[]++[]}&
\vb{qrev [] []}&
\vb{qrev (rev xs) []}\\
\vb{qrev [] (rev xs)}&
\vb{qrev (rev xs) ys}&
\vb{qrev [] xs}&
\vb{qrev xs []}\\
\vb{[]++qrev xs ys}&
\vb{qrev [] (xs++ys)}&
\vb{(x:xs)++[]}&
\vb{qrev xs ys++[]}\\
\vb{qrev (x:[]) xs}&
\vb{qrev [] (x:xs)}&
\vb{rev []}&
\vb{rev (qrev ys xs)}\\
\vb{rev (rev xs)}&
\vb{[]++rev xs}&
\vb{rev xs}&
\vb{rev xs++ys}\\
\vb{xs}&
\vb{[]++xs}&
\vb{xs++[]}&
\vb{(xs++ys)++[]}\\
\vb{[]++(xs++ys)}&
\vb{xs++ys}&
\vb{(x:[])++xs}&
\vb{xs++(x:[])}
\end{tabular}

\end{frame}

\begin{frame}[fragile]
\frametitle{Equivalence classes partitioning}

\makebox[\textwidth][c]{
\begin{tikzpicture}

\node at (0,2) [eqblock,text width=90] (src) {
\vb{xs}\\
\vb{xs++[]}\\
\vb{[]++xs}\\
\vb{qrev [] xs}\\
\vb{rev (rev xs)}\\
\vb{qrev (rev xs) []}\\
};


\node at (0,-2) [eqblock,text width=90] {
\vb{xs++ys}\\
\vb{qrev (rev xs) ys}\\
\vb{[]++(xs++ys)}\\
\vb{qrev [] (xs++ys)}\\
\vb{(xs++ys)++[]}\\
};

\node at (4,2) [eqblock,text width=60] {
\vb{[]}\\
\vb{rev []}\\
\vb{qrev [] []}\\
\vb{[]++[]}\\
};

\node at (4,-2)  [eqblock,text width=80] {
\vb{x:xs}\\
\vb{[]++(x:xs)}\\
\vb{qrev [] (x:xs)}\\
\vb{(x:xs)++[]}\\
\vb{(x:[])++xs}\\
\vb{qrev (x:[]) xs}\\
};

\node at (8,2)  [eqblock,text width=120] {
\vb{qrev xs ys}\\
\vb{rev (qrev ys xs)}\\
\vb{rev xs++ys}\\
\vb{[]++qrev xs ys}\\
\vb{qrev [] (qrev xs ys)}\\
\vb{qrev xs ys++[]}\\
\vb{qrev (qrev ys xs) []}\\

};

\node at (8,-2)  [eqblock,text width=100] {
\vb{rev xs}\\
\vb{qrev xs []}\\
\vb{[]++rev xs}\\
\vb{qrev [] (rev xs)}\\
};

\end{tikzpicture}
}

\end{frame}

\begin{frame}[fragile]
  \frametitle{Example of pruned equations from QuickSpec}

\begin{verbatim}
Universe has 2893 terms, 1824 classes
== equations ==
 1: xs++[] == xs
 2: qrev xs [] == rev xs
 3: []++xs == xs
 4: qrev [] xs == xs
 5: (x:xs)++ys == x:(xs++ys)
 6: (xs++ys)++zs == xs++(ys++zs)
 7: qrev xs ys++zs == qrev xs (ys++zs)
 8: qrev (x:xs) ys == qrev xs (x:ys)
 9: qrev (xs++ys) zs == qrev ys (qrev xs zs)
10: qrev (qrev xs ys) zs == qrev ys (xs++zs)
\end{verbatim}

\end{frame}

\begin{frame}
  \frametitle{Overview of HipSpec}
  \def\hltrans{} % Highlight translation
  \newcommand{\h}[2]{\action<#1->{#2}}

% Define block styles
\tikzstyle{bigblock}   = [rectangle, draw=Purpleee, thick,
                          text width=4.75em, text centered,
                          minimum height=10em,
                          rounded corners=.8ex]

\tikzstyle{smallblock} = [rectangle,
                          text width=6em, text centered,
                          minimum height=3em,thick,
                          draw=Purpleee,rounded corners=.8ex]

\tikzstyle{arr} = [->,>=stealth',semithick]

\makebox[\textwidth][c]{
\begin{tikzpicture}

    % Place nodes
    \node at (-0.75,0)     [bigblock] (src) {Program};
    \pause

    \node at (4,1.5)  [smallblock] (conj) {Conjectures};
    \draw [arr] (src)  to [bend left=15] node [above]
       {
         QuickSpec
       } (conj);

    \pause
    \node at (4,-1.5) [smallblock] (thy)  {First-Order Theory};
    \draw [arr] (src)  to [bend left=-15] node [below,text centered,text width=5em]
       {
         Translation (Hip)
       } (thy);
    \pause
    \node at (8.5,0)     [smallblock,text width=8em] (atp)  {Theorem Prover \\ {\small (E, Vampire, Z3...)}};

    \node at (4.5,0) (text) {Induction (Hip)};


    \draw [arr] (conj) to [bend right=17]  (atp);
    \draw [arr] (thy)  to [bend right=-17] (atp);

    \pause
    \draw [arr] (atp)  to [bend left=15]
        node [right,near start] {\texttt{ Theorem}}
        node [below right,very near end] {Extend theory} (thy);

    \pause
    \draw [arr] (atp)  to [bend left=-15]
        node [right,near start] {\texttt{ Timeout}}
        node [above right,very near end] {Open conjecture} (conj);

\end{tikzpicture}
}



  \undef\hltrans
\end{frame}

\newcommand\fn[1]{\mathrm{#1}}
\newcommand\fcons   [2]{\fn{cons}(#1,#2)}
\newcommand\frev    [1]{\fn{rev}(#1)}
\newcommand\fxs     [0]{\textsf{xs}}
\newcommand\fx      [0]{\textsf{x}}
\newcommand\fys     [0]{\textsf{ys}}
\newcommand\fy      [0]{\textsf{y}}
\newcommand\fappend [2]{\fn{append}(#1,#2)}
\newcommand\fnil    [0]{\fn{nil}}

\begin{frame}[fragile]
  \frametitle{Hip : The Haskell Inductive Prover}

  \begin{itemize}
    \item Translates the Haskell source definitions to first order logic
\up
  \end{itemize}
  \begin{verbatim}
      rev (x:xs) = rev xs ++ [x]
      rev []     = []
  \end{verbatim}
\up
Function definition axioms:
\begin{align*}
1 && \faa{\fx}{\fxs} & \frev{\fcons{\fx}{\fxs}} = \fappend{\frev{\fxs}}{\fcons{\fx}{\fnil}} \\
2 &&                 & \frev{\fnil}             = \fnil
\end{align*}
Data type axioms:
\begin{align*}
3 && \faaaa{\fx}{\fxs}{\fy}{\fys} & \fcons{\fx}{\fxs} = \fcons{\fy}{\fys} \implies \fx = \fy \land \fxs = \fys \\
4 && \faa{\fx}{\fxs} & \fnil \neq \fcons{\fx}{\fxs}
\end{align*}
\up
  \begin{itemize}
  \item Also supports higher-order functions and partial application
  \item Applies structural induction on properties
  \end{itemize}

\end{frame}

\begin{frame}[fragile]
  \frametitle{Picking a conjecture, and the main loop}

  \def\onlyloop{}
  \newcommand{\h}[2]{\action<#1->{#2}}

% Define block styles
\tikzstyle{bigblock}   = [rectangle, draw=Purpleee, thick,
                          text width=4.75em, text centered,
                          minimum height=10em,
                          rounded corners=.8ex]

\tikzstyle{smallblock} = [rectangle,
                          text width=6em, text centered,
                          minimum height=3em,thick,
                          draw=Purpleee,rounded corners=.8ex]

\tikzstyle{arr} = [->,>=stealth',semithick]

\makebox[\textwidth][c]{
\begin{tikzpicture}

    % Place nodes
    \node at (-0.75,0)     [bigblock] (src) {Program};
    \pause

    \node at (4,1.5)  [smallblock] (conj) {Conjectures};
    \draw [arr] (src)  to [bend left=15] node [above]
       {
         QuickSpec
       } (conj);

    \pause
    \node at (4,-1.5) [smallblock] (thy)  {First-Order Theory};
    \draw [arr] (src)  to [bend left=-15] node [below,text centered,text width=5em]
       {
         Translation (Hip)
       } (thy);
    \pause
    \node at (8.5,0)     [smallblock,text width=8em] (atp)  {Theorem Prover \\ {\small (E, Vampire, Z3...)}};

    \node at (4.5,0) (text) {Induction (Hip)};


    \draw [arr] (conj) to [bend right=17]  (atp);
    \draw [arr] (thy)  to [bend right=-17] (atp);

    \pause
    \draw [arr] (atp)  to [bend left=15]
        node [right,near start] {\texttt{ Theorem}}
        node [below right,very near end] {Extend theory} (thy);

    \pause
    \draw [arr] (atp)  to [bend left=-15]
        node [right,near start] {\texttt{ Timeout}}
        node [above right,very near end] {Open conjecture} (conj);

\end{tikzpicture}
}



  \undef\onlyloop

  \begin{enumerate}
    \item Try to prove ``smallest'' unproved equation this round
    \item Failure: save this for next round
    \item Success: extend the theory
    \item When a round did not lead to any successes, or everything proved, terminate.
  \end{enumerate}

  \pause

  We use light-weight reasoning by means of a congruence closure to
  prune away conjecture that can be proved without induction. % equations from QuickSpec holding up to equality.

\end{frame}

\begin{frame}
  \frametitle{}
    \begin{center}
    {\color{Purpleee} \Huge Demo!}
    \end{center}
\end{frame}

\begin{frame}[fragile]
  \frametitle{Evaluation - First Test Suite}

  First suite from
  \emph{Case-Analysis for Rippling and Inductive Proof}
  \hspace{1em} by Johansson, Dixon and Bundy (2010)

  \dn

  85 conjectures, 71 equational.

  \pause

  \begin{center}
  \begin{tabular}{l l}
    Tool       & Proved conjectures (of 85) \\
    \hline
    Zeno       & 82 \\[3pt]
    ACL2s      & 74 \\[3pt]
    IsaPlanner & 47 \\[3pt]
    Dafny      & 45 \\[3pt]
    HipSpec    & 67 (of 71)
  \end{tabular}
  \end{center}

  \pause

  \invisible<1-2>{
     %Buggy!? wtf latex
     \only<1-3>{
       Unproved:
       \up
       \begin{center}
       \begin{tabular}{>{\small}l >{\small}l}
       \hs{count n xs = count n (sort xs)}, & \hs{len (filter p xs) <= len xs } \\
       \hs{sorted (sort xs) = True},        & \hs{len (delete n xs) <= len xs }
       \end{tabular}
       \end{center}
       But they require conditional lemmas!
     }
     \only<4>{
       Two properties only proved by HipSpec!
       \begin{center}
       \begin{tabular}{>{\small}l}
       \hs{rev (drop i xs) = take (len xs - i) (rev xs)} \\
       \hs{rev (take i xs) = drop (len xs - i) (rev xs)}
       \end{tabular}
       \end{center}
       Requires a bunch of quite far-fetched lemmas
     }
  }
\end{frame}

\begin{frame}[fragile]
  \frametitle{Evaluation - Second Test Suite}

  Second test suite from
      \emph{Productive Use of Failure in Inductive Proof}
      \hspace{1em} by Bundy and Ireland (1995)

  \dn

  Their tool CLAM supposedly proves all, but some properties contrived
  towards their tool, cf \hs{rev (rev xs ++ []) = xs}

  \dn

  49 theorems, 38 equational. \pause HipSpec proves 36!

  \pause
  \dn

  Unproved:
  \begin{center}
  \begin{tabular}{>{\small}c >{\small}l}
  No  & Conjecture \\
  \hline
  T14 & \verb!ordered (isort xs) = True! \\
  T50 & \verb!count x (isort xs) = count x xs! \\
  \end{tabular}
  \end{center}

  \dn

  Zeno? \pause Proves 21/49

\end{frame}

\begin{frame}
  \frametitle{}
    \begin{center}
    {\color{Purpleee} \Huge Success!\pause?}

    \vspace{\baselineskip}

    There might be some limitations... ;)
    \end{center}
\end{frame}

\begin{frame}[fragile]
  \frametitle{Future work and current limitations}

  \begin{itemize}
    \item Better heuristics (Equation order)
    \item Big theories and scalability
    \item Conditional properties
    \item Non-terminating programs and infinite values
  \end{itemize}

\end{frame}

\begin{frame}[fragile]
  \frametitle{Conclusion}

  Exploring the laws that hold through testing does not only help your
  understanding, but also helps to prove properties.

  \dn

  A form of completeness from QuickSpec: If there are laws up to a
  certain term size then QuickSpec is guaranteed to find them.

  \dn

  \emph{If the lemma is there, HipSpec will eventually try to prove it!}

\end{frame}

\begin{frame}[fragile]
  \frametitle{Extra slides}
\end{frame}

\begin{frame}[fragile]
  \frametitle{Obtaining HipSpec}

  \begin{itemize}

    \item Clone the repository:
      \\ \hs{git clone http://github.com/danr/hipspec}

    \item Installation (requires GHC):
       \\ \hs{cd hipspec}
       \\ \hs{git submodule update --init}
       \\ \hs{cabal install}

    \item Install a theorem prover (say eprover)

    \item Try an example!
       \\ \hs{cd testsuite/}
       \\ \hs{runghc Reverse.hs}

  \end{itemize}

  % These slides:
  % \hs{http://web.student.chalmers.se/\~{}danr/hipspec-slides.pdf}

\end{frame}
\begin{frame}[fragile]

  \frametitle{Future work: Big theories}

  Taking all your functions from a big program:

  \begin{itemize}
    \item Testing takes a long time
    \item Lemmas become unrelated
  \end{itemize}

  \pause

  \vspace{\baselineskip}

  How do we know when functions are related?

  \begin{equation*}
    \hs{length (xs ++ ys)} = \hs{length (ys ++ xs)}
  \end{equation*}

\end{frame}

\begin{frame}[fragile]
  \frametitle{Future work: Conditional properties}

  Lemmas with implications:

  \begin{equation*}
    \hs{sorted xs} = \hs{True} \implies \hs{sorted (insert x xs)} = \hs{True}
  \end{equation*}

  \pause
  {\color{Purpleee} A trick:} use a new data type, abstract for HipSpec:

  \begin{verbatim}
data SortedList = SortedList { getSortedList :: [Nat] }

instance Arbitary SortedList where
  arbitrary = SortedList . scanl1 (+) `fmap` arbitrary
  \end{verbatim}

  \vspace{-\baselineskip}

  \pause
  Now, we can state the property in terms of a sorted list \hs{sl}:

  \begin{equation*}
    \hs{sorted (insert x (getSortedList sl))} = \hs{True}
  \end{equation*}

  Need a notation to HipSpec that \hs{SortedList} has
  a \hs{sorted} invariant.


\end{frame}



% \begin{frame}[fragile]
%   \frametitle{Related work}
%
%   {\color{red} incomplete slide}
%
%   \begin{itemize}
%
%     \item Syntactic inductive approaches (rippling)
%
%     \item IsaCosy
%
%     \item Zeno
%
%   \end{itemize}
% \end{frame}


\begin{frame}[fragile]
  \frametitle{Proof: \Large\texttt{rev (drop i xs) = take (len xs-i) (rev xs)}}
  \begin{center}
  \begin{tabular}{>{\small}c >{\small}l}
   No & Conjecture \\
   \hline
  1 & \verb!len (drop x xs) = len xs-x!               \\[-1pt]
  2 & \verb!len xs          = len (rev xs)!           \\[-1pt]
  3 & \verb!xs              = take x xs++drop x xs!   \\[-1pt]
  4 & \verb!rev (ys++xs)    = rev xs++rev ys!         \\[-1pt]
  5 & \verb!xs              = take (len xs) (xs++ys)!
  \end{tabular}
  \end{center}

  \begin{align*}
  & \hspace{-2em} \small \hs{rev (drop i xs)}                                                   & \small = \left\{ 5 \right\} \\[-3pt]
  & \hspace{-2em} \small \hs{take (len (rev (drop i xs))) (rev (drop i xs)++rev (take i xs))}   & \small = \left\{ 2 \right\} \\[-3pt]
  & \hspace{-2em} \small \hs{take (len (drop i xs)) (rev (drop i xs)++rev (take i xs))}         & \small = \left\{ 1 \right\} \\[-3pt]
  & \hspace{-2em} \small \hs{take (len xs-i) (rev (drop i xs)++rev (take i xs))}                & \small = \left\{ 4 \right\} \\[-3pt]
  & \hspace{-2em} \small \hs{take (len xs-i) (rev (take i xs++drop i xs))}                      & \small = \left\{ 3 \right\} \\[-3pt]
  & \hspace{-2em} \small \hs{take (len xs-i) (rev xs)}
  \end{align*}
\end{frame}

\begin{frame}[fragile]
  \frametitle{Future work: Conditional properties II}

  What about

  \begin{equation*}
    \hs{x < y} = \hs{True} \land
    \hs{y < z} = \hs{True} \implies
    \hs{x < z} = \hs{True}
  \end{equation*}

  \pause
  Same trick?

  \begin{verbatim}
    data Pair = Pair { smaller :: Nat , larger :: Nat }
  \end{verbatim}

  Can we state the property?

  \begin{equation*}
    \hs{smaller p1 < larger p2} = \hs{True}
  \end{equation*}

  \pause
  {\color{red} Problem:} how are \hs{p1} and \hs{p2} related?

\end{frame}

\begin{frame}[fragile]
  \frametitle{Limitation: Expensive calculations}

  Imagine a program which does exponentiation, \hs{**}, on
  unary nats \hs{data Nat = Zero | Succ Nat}

  \vspace{\baselineskip}

  Too expensive to caluclate \hs{x ** (y ** z)}.

  \vspace{\baselineskip}

  \pause

  \begin{verbatim}
  f x y = if y == 1000000000000
               then 0
               else x + y
  \end{verbatim}
\end{frame}


%
%%
%%
%%\begin{frame}
%%  \frametitle{Behind the Scenes}
%%%  \begin{itemize}
%%%    \item Translation to first order logic\\
%%%    \item Trying different induction techniques\\
%%%    \item Running automated theorem provers\\
%%%  \end{itemize}
%%\end{frame}
%%
%%
%%\begin{frame}[fragile]
%%\frametitle{Structural Induction Example}
%%
%%\begin{mathpar}
%%  \inferrule*
%%    {
%%      P(\leaf{x})
%%      \and
%%      P(u)\wedge P(v)\rightarrow P(\fork{u}{v})
%%    }
%%    { P(t) }
%%\end{mathpar}
%%
%%
%%\begin{frame}[fragile]
%%\frametitle{Future Work: Lemma synthesis from QuickSpec}
%%
%%\begin{verbatim}
%%1: return x := x:[]
%%2: x:xs := return x++xs
%%== equations ==
%%1: x:[] == return x
%%2: xs++[] == xs
%%3: []++xs == xs
%%4: reverse [] == []
%%5: (x:xs)++ys == x:(xs++ys)
%%6: (xs++ys)++zs == xs++(ys++zs)
%%7: reverse (return x) == return x
%%8: reverse (reverse xs) == xs
%%9: reverse xs++return x == reverse (x:xs)
%%10: reverse xs++reverse ys == reverse (ys++xs)
%%\end{verbatim}
%%\end{frame}
%%
%%%% MOVE TO END
%%\begin{frame}
%%  \frametitle{Related Work}
%%  \begin{itemize}
%%    \item Zeno (Sonnex, Drossopoulou, Eisenbach) {\footnotesize(Imperial College)}
%%
%%    \item Combining Interactive and Automatic Reasoning about Functional Programs \\
%%          (Bove, Dybjer, Sicard)
%%    \item Dependently Typed Programming based on Automated Theorem Proving \\
%%          (Armstrong, Foster, Struth) {\footnotesize(Univ. of Sheffield)}
%%  \end{itemize}
%%\end{frame}
%%
%%\begin{frame}
%%\frametitle{Conclusion}
%%
%%\begin{itemize}
%%\item FOL is expressible enough for various Haskell concepts; pattern
%%  matching, infinite data structures and higher order functions
%%\item Thanks to referential transparency
%%\item Possible to prove many properties without proving termination
%%\end{itemize}
%%\end{frame}
%%
%
%
%
%
%%%%%% Old macros
%
%%\makeatletter
%%\newcommand*{\rom}[1]{\text{\footnotesize\expandafter\@slowromancap\romannumeral #1@.}}
%%\newcommand*{\romnodot}[1]{\text{\footnotesize\expandafter\@slowromancap\romannumeral #1@}}
%%\makeatother
%
%%\newcommand\note[1]{\mbox{}\marginpar{\footnotesize\raggedright\hspace{0pt}\emph{#1}}}
%%\newcommand\note[1]{}
%%\newcommand\PA{\mathcal{P\!A}}
%%\newcommand\hs[1]{\verb~#1~}
%%\newcommand\ts[1]{\verb~#1~}
%%\newcommand\fn[1]{#1}
%%\newcommand\ptr[1]{\fn{\operatorname{#1-ptr}}}
%%\newcommand\appfn{@}
%%\newcommand\app[2]{#1 \, \appfn \, #2}
%%\newcommand\appp[2]{(#1) \, \appfn \, #2}
%%\newcommand\ex[1]{\exists \, #1 \, . \,}
%%\newcommand\nexxx[3]{\nexists \, #1 , #2 , #3 . \,}
%%
%%\newcommand{\HRule}{\rule{\linewidth}{0.5mm}}%\usetikzlibrary {\trees,positioning,arrows}
%%
%%\newcommand\fixb[0]{ ^{\bullet}}
%%\newcommand\fixw[0]{ ^{\circ}}
%%
%%\newcommand\tofix[1]{#1^{\bullet}}
%%\newcommand\unfix[1]{#1^{\circ}}
%%
%%\newcommand\append[0]{\texttt{\small{++}}}
%%
%%\newcommand{\xsys}[2]{#1 \, xs \, #2 & = #1 \, ys #2}
%%\newcommand{\desca}[1]{  & \hspace{44.5mm}                            \{ \text{#1} \}}
%%\newcommand{\descra}[1]{ & \hspace{35mm} \Rightarrow     \hspace{4mm} \{ \text{#1} \}}
%%\newcommand{\descla}[1]{ & \hspace{35mm} \Leftarrow      \hspace{4mm} \{ \text{#1} \}}
%%\newcommand{\desclra}[1]{& \hspace{35mm} \Leftrightarrow \hspace{4mm} \{ \text{#1} \}}
%%
%%\newcommand\lub[1]{\sqcup_{#1}}
%%
%%\newcommand\defof[1]{definition of #1}
%%
%%\newcommand\w[0]{\,\,}
%%\newcommand\eq[0]{ = }
%%
%%\newcommand{\defBNF}[4] {\text{#1}\quad&#2&::=&\;#3&\text{#4}}
%%\newcommand{\defaltBNF}[2] {&&|&\;#1&\text{#2}}
%%
%%\newcommand{\hstup}[2]{\hs{(} #1 \hs{,} #2 \hs{)}}
%%
%%\newcommand{\nsqsubseteq}{\,\,\, /\!\!\!\!\!\!\sqsubseteq}
%%
%%\newcommand{\bindname}{>\!\!>\!\!=}
%%\newcommand{\bind}[2]{#1 \bindname #2}
%%\newcommand{\bindp}[3]{\fn{bind'}(#1,#2,#3)}
%%\newcommand{\fork}[2]{\fn{fork}(#1,#2)}
%%\newcommand{\forkr}[1]{\fn{right}(#1)}
%%\newcommand{\forkl}[1]{\fn{left}(#1)}
%%\newcommand{\leaf}[1]{\fn{leaf}(#1)}
%%\newcommand{\unleaf}[1]{\fn{unleaf}(#1)}
%%\newcommand{\return}[1]{\fn{return}(#1)}
%%
%%\newcommand{\bindb}[2]{#1 \tofix{\bindname} #2}
%%\newcommand{\bindw}[2]{#1 \unfix{\bindname} #2}
%%
%%
%%\newcommand\Inf{\fn{Inf}}
%%\newcommand\Total{\fn{Total}}
%%\newcommand\Fin{\fn{Fin}}

\end{document}
