%!TEX TS-program = xelatex
%!TEX encoding = UTF-8 Unicode

\documentclass[serif,professionalfont]{beamer}
\def\maindocument{} % To tell tikz images that they are not stand alone

\usepackage{tikz}
\usetikzlibrary{shapes,arrows,calc}
\usepackage{verbatim}

% \usepackage[utf8]{inputenc}
% \usepackage[T1]{fontenc}
%\usepackage{fixltx2e}
%\usepackage{graphicx}
%\usepackage{longtable}
%\usepackage{float}
%\usepackage{wrapfig}
%\usepackage{soul}
%
%\usepackage{textcomp}
%\usepackage{marvosym}
%\usepackage{wasysym}
%\usepackage{latexsym}
\usepackage{amssymb}

\usepackage{hyperref}

\usepackage{mathpartir}
\usepackage{color}
%\tolerance=1000
\usepackage{inconsolata}
\usepackage{amsmath}
\usepackage{array}
\providecommand{\alert}[1]{\textbf{#1}}

\definecolor{Purpleee}{RGB}{140,20,140}
\definecolor{PurpleL}{RGB}{255,170,255}
\definecolor{MyGreen}{RGB}{5,95,5}
\setbeamercolor{title}{fg=Purpleee}
\setbeamercolor{frametitle}{fg=Purpleee}
\setbeamercolor{structure}{fg=Purpleee}


\TeXXeTstate=1
\usepackage{mathspec,xltxtra,xunicode}
% \setsansfont{Gill Sans}

% \setmainfont[Scale=1]{Gill Sans}
%\setmonofont[Scale=0.8]{Monaco}
\setmonofont{Inconsolata}

% \setmathsfont(Digits,Latin,Greek){Gill Sans}

\usefonttheme[onlymath]{serif}

%\usepackage{fontspec}
%\defaultfontfeatures{Mapping=tex-text}
%\setsansfont[Ligatures={Common}]{Futura}


\usepackage{listings}

\lstnewenvironment{codex}[1][]%
  {\noindent
   \minipage{\linewidth}
   \vspace{0.2\baselineskip}
%   \vspace{-0.4\baselineskip}
   \lstset{basicstyle=\ttfamily,
%           frame=single,
           language=Haskell,
           keywordstyle=\color{black},
           #1}}
  {%\vspace{-0.8\baselineskip}
   \endminipage}

\title{{\Huge HipSpec}}
\subtitle{{\LARGE Automating Inductive Proofs \\ using Theory Exploration}}
\institute{Chalmers University of Technology}

\author{Dan Ros\'en
     \vspace{\baselineskip} \\
     Koen Claessen, Moa Johansson, Nicholas Smallbone}
\date{May 31, 2013}

\newcommand\fa[1]{ \forall \, #1 . \,}
\newcommand\faa[2]{ \forall \, #1 , #2 . \,}
\newcommand\faaa[3]{ \forall \, #1 , #2 , #3 . \,}
\newcommand\faaaa[4]{ \forall \, #1 , #2 , #3 , #4 . \,}
\newcommand\up[0]{\vspace{-\baselineskip}}
\newcommand\dn[0]{\vspace{\baselineskip}}
\newcommand\hs[1]{\texttt{#1}}

\newcommand\x[0]{\hs{x}}
\newcommand\xs[0]{\hs{xs}}
\newcommand\ys[0]{\hs{ys}}
\newcommand\xxs[0]{\hs{x:xs}}
\newcommand\nil[0]{\hs{[]}}
\newcommand\p[1]{\textsf{P}(#1)}

\newcommand{\highlight}[1]{\colorbox{PurpleL}{\ensuremath{#1}}}

\begin{document}

\maketitle

\defverbatim{\rotatedef}{%
\begin{verbatim}
    rotate Z     xs     = xs
    rotate (S n) []     = []
    rotate (S n) (x:xs) = rotate n (xs ++ [x])
\end{verbatim}
}

\defverbatim{\rotateex}{%
\begin{verbatim}
    rotate 1 [1,2,3,4] = [2,3,4,1]
    rotate 2 [1,2,3,4] = [3,4,1,2]
    rotate 3 [1,2,3,4] = [4,1,2,3]
    rotate 4 [1,2,3,4] = [1,2,3,4]
\end{verbatim}
}

\defverbatim{\rotateprop}{%
\begin{equation*}
    \fa{\xs} \hs{rotate (length xs) xs} = \hs{xs}
\end{equation*}
}

\defverbatim{\rotategen}{%
\begin{equation*}
    \faa{\xs}{\ys} \hs{rotate (length (xs ++ ys)) (xs ++ ys)} = \hs{ys ++ xs}
\end{equation*}
}

\defverbatim{\rotatestep}{%
\begin{align*}
    \hs{rotate (length (x:xs)) (x:xs) } & = \\
    \hs{rotate (S (length xs)) (x:xs) } & = \\
    \hs{rotate (length xs) (xs ++ [x])} & =
\end{align*}
}

\begin{frame}[fragile]
    \frametitle{Rotate example}

    \begin{overprint}
    \rotatedef

    \onslide<1> \rotateex

    \onslide<2> \rotateex \rotateprop

    \onslide<3> \rotateprop

    \onslide<4> \rotateprop \rotatestep

    \onslide<5> \rotateprop \rotatestep
    \begin{center}
      {\color{red} Stuck!}
    \end{center}

    \end{overprint}

\end{frame}

% Rotate demo

% We're doing theory exploration right now, to conjecture useful lemmas,
% done by a tool called QuickSpec,
% Now we've started proving things by induction, and these properties
% were not specified in the source code but were conjectured during TE
% oh there's our property being proved! hooray!

\begin{frame}
    \frametitle{HipSpec vs Rotate}

    \rotategen

    \pause
    \begin{center}
    (also requires associativity and right identity of \hs{++})
    \end{center}

\end{frame}

\begin{comment} % old setting
  Prove properties of functional programs using rewriting and
  induction.

  {\color{Purpleee} Problem:} Som properties require lemmas to be proved, that

  \begin{itemize}
    \item Needs to be conjectured,
    \item Requires induction to be proved, and
    \item Might require lemmas themselves
  \end{itemize}
\end{comment}


\begin{frame}
    \frametitle{QuickSpec: the Theory Exploration Phase}
    Generates a bunch of terms: % well-typed
    \dn
    
\begin{comment}
xs,xs++[],[]++xs,rot Z xs,rot (len xs) xs

xs++ys,[]++(xs++ys),rot Z (xs++ys),(xs++ys)++[],rot (len ys) (ys++xs)

xs++(ys++ys),(xs++ys)++ys

rot n (xs++xs),rot n xs++rot n xs

len xs,len (rot n xs),len (rot m xs),len (rot o xs)

len (xs++ys),len (ys++xs)

rot n (rot m xs),rot m (rot n xs)

maybe:
[],rot n [],rot m [],rot o [],[]++[],rot Z [],rot (len xs) [],rot (len ys) [],rot (len zs) [],rot (S n) [],rot (S m) [],rot (S o) [],rot (S Z) []



xs
xs++[]
[]++xs
rot Z xs
rot (len xs) xs

xs++ys
[]++(xs++ys)
rot Z (xs++ys)
(xs++ys)++[]
rot (len ys) (ys++xs)

xs++(ys++ys)
(xs++ys)++ys

rot n (xs++xs)
rot n xs++rot n xs

len xs
len (rot n xs)

len (xs++ys)
len (ys++xs)

rot n (rot m xs)
rot m (rot n xs)
\end{comment}

\newcommand\vb[1]{\makebox[3.5cm][l]{\tt #1}}

%\begin{tabular}{>{\footnotesize}l >{\footnotesize}l >{\footnotesize}l >{\footnotesize}l}
\vb{rot (len xs) xs}
\vb{len xs}
\vb{xs++(ys++ys)}
\vb{rot n (xs++xs)}
\vb{rot n (rot m xs)}
\vb{rot n xs++rot n xs}
\vb{(xs++ys)++ys}
\vb{rot Z (xs++ys)}
\vb{rot m (rot n xs)}
\vb{xs}
\vb{len (rot m xs)}
\vb{len (rot n xs)}
\vb{xs++ys}
\vb{len (ys++xs)}
\vb{len (rot o xs)}
\vb{rot Z xs}
\vb{len (xs++ys)}
\vb{[]++xs}
\vb{(xs++ys)++[]}
\vb{xs++[]}
\vb{rot (len ys) (ys++xs)}
%\end{tabular}

\end{frame}

\begin{frame}
    \frametitle{Partitioning into Equivalence Classes}

    \defverbatim{\cla}{%
\begin{verbatim}
xs
xs++[]
[]++xs
rot Z xs
rot (len xs) xs
\end{verbatim}
}

\defverbatim{\clb}{%
\begin{verbatim}
xs++ys
[]++(xs++ys)
rot Z (xs++ys)
(xs++ys)++[]
rot (len ys) (ys++xs)
\end{verbatim}
}

\defverbatim{\clc}{%
\begin{verbatim}
xs++(ys++ys)
(xs++ys)++ys
\end{verbatim}
}

\defverbatim{\cld}{%
\begin{verbatim}
rot n (xs++xs)
rot n xs++rot n xs
\end{verbatim}
}

\defverbatim{\cle}{%
\begin{verbatim}
len xs
len (rot n xs)
\end{verbatim}
}

\defverbatim{\clf}{%
\begin{verbatim}
len (xs++ys)
len (ys++xs)
\end{verbatim}
}

\defverbatim{\clg}{%
\begin{verbatim}
rot n (rot m xs)
rot m (rot n xs)
\end{verbatim}
}


\tikzstyle{eqblock} = [rectangle, draw=Purpleee, thick, text width=4.75em]
\newcommand\vb[1]{\mbox{{\tt #1}}}
\makebox[\textwidth][c]{%
\begin{tikzpicture}

\node at (0.5,2) [eqblock,text width=85] {\cla};

\node at (0.5,-2) [eqblock,text width=115] {\clb};

\node at (6,3) [eqblock,text width=65] {\clc} ;

\node at (6,1)  [eqblock,text width=100] {\cld};

\node at (8,-1)  [eqblock,text width=80] {\cle};

\node at (4.5,-1)  [eqblock,text width=70] {\clf};

\node at (6.5,-3)  [eqblock,text width=90] {\clg};

\end{tikzpicture}
}



    % read off the equations from the classes

    % completeness up to some depth

    % false properties are not a problem given that
    % the function definitions do not have tricky corner cases
    % that the random value generators do not cover
\end{frame}

\begin{frame}[fragile]
  \frametitle{Hip: The Haskell Inductive Prover}

  \begin{itemize}
    \item Translate to typed first order logic
          % Translates the function definitions to typed first order logic
    \item Apply structural induction
          % Applies structural induction on properties
  \end{itemize}

  Also supports higher-order functions and partial application
\end{frame}

\begin{frame}
  \frametitle{Overview of HipSpec}
  \newcommand{\h}[2]{\action<#1->{#2}}

% Define block styles
\tikzstyle{bigblock}   = [rectangle, draw=Purpleee, thick,
                          text width=4.75em, text centered,
                          minimum height=10em,
                          rounded corners=.8ex]

\tikzstyle{smallblock} = [rectangle,
                          text width=6em, text centered,
                          minimum height=3em,thick,
                          draw=Purpleee,rounded corners=.8ex]

\tikzstyle{arr} = [->,>=stealth',semithick]

\makebox[\textwidth][c]{
\begin{tikzpicture}

    % Place nodes
    \node at (-0.75,0)     [bigblock] (src) {Program};
    \pause

    \node at (4,1.5)  [smallblock] (conj) {Conjectures};
    \draw [arr] (src)  to [bend left=15] node [above]
       {
         QuickSpec
       } (conj);

    \pause
    \node at (4,-1.5) [smallblock] (thy)  {First-Order Theory};
    \draw [arr] (src)  to [bend left=-15] node [below,text centered,text width=5em]
       {
         Translation (Hip)
       } (thy);
    \pause
    \node at (8.5,0)     [smallblock,text width=8em] (atp)  {Theorem Prover \\ {\small (E, Vampire, Z3...)}};

    \node at (4.5,0) (text) {Induction (Hip)};


    \draw [arr] (conj) to [bend right=17]  (atp);
    \draw [arr] (thy)  to [bend right=-17] (atp);

    \pause
    \draw [arr] (atp)  to [bend left=15]
        node [right,near start] {\texttt{ Theorem}}
        node [below right,very near end] {Extend theory} (thy);

    \pause
    \draw [arr] (atp)  to [bend left=-15]
        node [right,near start] {\texttt{ Timeout}}
        node [above right,very near end] {Open conjecture} (conj);

\end{tikzpicture}
}



  \pause
  \begin{itemize}
    \item Try to prove ``smallest'' unproved equation
    \item Terminate everything is proved (or when the current theory cannot prove any more open conjectures)
  \end{itemize}
\end{frame}

% here we can contrast with other kinds of theory exploration
% system

\begin{frame}
  \frametitle{Rotate, revisited}

    \rotateprop \rotatestep
    \begin{center}
      {\color{red} Stuck!}
    \end{center}

\end{frame}

% "So the tools IsaCosy and IsaScheme can conjecture a theory,
% but not fully automatically when proving a user-stated property,
% which is what HipSpec can. But HipSpec can also be run without
% specifying what to prove! Let's have a look at the output
% of when rotate was run"

% See, the property we wanted to prove was conjectured by quickspec.
% And of course it is! It is within the depth limit, so this is fully
% expected. So we could run this example without specifying that this
% is the property we want to show, and it will still prove it!
% But instead of looking at that, let's look at another common
% problem in this literature, namely rev/qrev...
% (explanation + BAM)
% Another example is this, so we have unary/peano nats and plus
% and multiplication defined over this. So we should expect
% that hipspec can prove that this is a commutative semiring...
% let's try...

\begin{frame}
  \frametitle{HipSpec the Theory Exploration System}

  % so this is what we just seen

  \begin{center}
      Saturate a theory and have it nicely presented
  \end{center}

  \dn
  \pause

  % other examples that we can do:

  \begin{itemize}
      \item \hs{data Integer = Positive Nat | Negative Nat}
      \item \hs{data BinNat  = Zero | ZeroAnd BinNat | OneAnd BinNat}
  \end{itemize}

  % so this is an interesting piece of work to hook this up
  % with an interactive theorem prover, isa/coq/agda
  % ... could make us spend less time on proving boring properties ;)

\end{frame}

\begin{frame}[fragile]
  \frametitle{Conjecturing Conditionals}

  \begin{verbatim}
    isort :: [Nat] -> [Nat]

    insert :: Nat -> [Nat] -> [Nat]

    sorted :: [Nat] -> Bool
  \end{verbatim}

  \begin{equation*}
    \fa{\xs}  \hs{sorted (isort xs)} = \hs{True}
  \end{equation*}

  % It's one property that we currently cannot prove fully automatically

  Requires:

  \begin{equation*}
    \fa{\xs}  \hs{sorted xs} = \hs{True} \Rightarrow
              \hs{sorted (insert x xs)} = \hs{True}
  \end{equation*}

  % How do we conjecture such properties?
  % Currently, QuickSpec is implemented at looking at equivalence classes that
  % are unconditionally true... would like to have some kind of precondition on
  % certain equivalence classes to talk about when xs is always sorted, for
  % instance. Another problem might be that it is so uncommon to generate
  % sorted lists, so we will probably need to incorporate other ongoing work on
  % generating random values satisfying some predicate.

\end{frame}

% how to jump to conclusions/rounding off here in a reasonable way?

\begin{frame}
  \frametitle{Conclusions}
  \begin{itemize}
    \item Evaluate your programs!
    \item Completeness up to a certain depth: \\
          \emph{If the lemma is there, HipSpec will eventually try to prove it!}
  \end{itemize}
\end{frame}

\begin{frame}
\frametitle{}
\begin{center}
\hs{github.com/danr/hipspec}
\end{center}
\end{frame}

\begin{frame}[fragile]
  \frametitle{What is HipSpec?}
\tikzstyle{block} = [rectangle, draw=Purpleee, thick, text width=4.75em, text centered]


\makebox[\textwidth][c]{\
\begin{tikzpicture}

  \node at (0,0) [block,text width=120] (src) {\
\\
\textbf{Haskell source}
{\small
\begin{align*}
&\hs{rev [] = []} \\
&\hs{rev (x:xs)} \\
&\quad\hs{= rev xs ++ [x]} \\ \\
&\hs{prop\_rev xs} \\
&\quad\hs{= rev (rev xs) =:= xs}
\end{align*}
}
};
%\pause
  \node at (5,2) [block,text width=120] (hip) {\
\textbf{Hip}
\\
\emph{Haskell Inductive Prover}
\begin{itemize}
\item FOL translation
\item Apply induction
\item Success\ % \pause
, or stuck!
\end{itemize}
};
%\invisible<1-3>{
  \node at (5,-2)  [block,text width=120] (qs) {\
\textbf{QuickSpec}
\\
Eq-theory from testing:
\vspace{-0.5\baselineskip}
{\small
\begin{align*}
&\hs{rev (xs ++ ys)} \\
&\quad\hs{= rev ys ++ rev xs} \\
&\hs{xs ++ [] = []} \\
&\hs{xs ++ (ys ++ zs) =} \\
&\quad\hs{(xs ++ ys) ++ zs}
\end{align*}
}
};

 % \onslide<5>{
    \node at (8.5,-2) [block] (lemmas)
       {\textbf{HipSpec} \\ \emph{Use these as lemmas!!}};
 % }

\end{tikzpicture}
%}
}

\end{frame}


\begin{comment}

\begin{frame}[fragile]
  \frametitle{Evaluation - Second Test Suite}

  Second test suite from
      \emph{Productive Use of Failure in Inductive Proof}
      \hspace{1em} by Bundy and Ireland (1995)

  \dn

  Their tool CLAM supposedly proves all, but some properties contrived
  towards their tool, cf \hs{rev (rev xs ++ []) = xs}

  \dn

  49 theorems, 38 equational. \pause HipSpec proves 36!

  \pause
  \dn

  Unproved:
  \begin{center}
  \begin{tabular}{>{\small}c >{\small}l}
  No  & Conjecture \\
  \hline
  T14 & \verb!ordered (isort xs) = True! \\
  T50 & \verb!count x (isort xs) = count x xs! \\
  \end{tabular}
  \end{center}

  \dn

  Zeno? \pause Proves 21/49

\end{frame}

\begin{frame}[fragile]
  \frametitle{Future work and current limitations}

  \begin{itemize}
    \item Better heuristics (Equation order)
    \item Big theories and scalability
    \item Conditional properties
    \item Non-terminating programs and infinite values
  \end{itemize}

\end{frame}


\begin{frame}[fragile]
  \frametitle{Conclusion}

  Exploring the laws that hold through testing does not only help your
  understanding, but also helps to prove properties.

  \dn

  A form of completeness from QuickSpec: If there are laws up to a
  certain term size then QuickSpec is guaranteed to find them.

  \dn

  \emph{If the lemma is there, HipSpec will eventually try to prove it!}

\end{frame}

\begin{frame}[fragile]
  \frametitle{Extra slides}
\end{frame}

\begin{frame}[fragile]
  \frametitle{Obtaining HipSpec}

  \begin{itemize}

    \item Clone the repository:
      \\ \hs{git clone http://github.com/danr/hipspec}

    \item Installation (requires GHC):
       \\ \hs{cd hipspec}
       \\ \hs{git submodule update --init}
       \\ \hs{cabal install}

    \item Install a theorem prover (say eprover)

    \item Try an example!
       \\ \hs{cd testsuite/}
       \\ \hs{runghc Reverse.hs}

  \end{itemize}

  % These slides:
  % \hs{http://web.student.chalmers.se/\~{}danr/hipspec-slides.pdf}

\end{frame}
\begin{frame}[fragile]

  \frametitle{Future work: Big theories}

  Taking all your functions from a big program:

  \begin{itemize}
    \item Testing takes a long time
    \item Lemmas become unrelated
  \end{itemize}

  \pause

  \vspace{\baselineskip}

  How do we know when functions are related?

  \begin{equation*}
    \hs{length (xs ++ ys)} = \hs{length (ys ++ xs)}
  \end{equation*}

\end{frame}

\begin{frame}[fragile]
  \frametitle{Future work: Conditional properties}

  Lemmas with implications:

  \begin{equation*}
    \hs{sorted xs} = \hs{True} \implies \hs{sorted (insert x xs)} = \hs{True}
  \end{equation*}

  \pause
  {\color{Purpleee} A trick:} use a new data type, abstract for HipSpec:

  \begin{verbatim}
data SortedList = SortedList { getSortedList :: [Nat] }

instance Arbitary SortedList where
  arbitrary = SortedList . scanl1 (+) `fmap` arbitrary
  \end{verbatim}

  \vspace{-\baselineskip}

  \pause
  Now, we can state the property in terms of a sorted list \hs{sl}:

  \begin{equation*}
    \hs{sorted (insert x (getSortedList sl))} = \hs{True}
  \end{equation*}

  Need a notation to HipSpec that \hs{SortedList} has
  a \hs{sorted} invariant.


\end{frame}



% \begin{frame}[fragile]
%   \frametitle{Related work}
%
%   {\color{red} incomplete slide}
%
%   \begin{itemize}
%
%     \item Syntactic inductive approaches (rippling)
%
%     \item IsaCosy
%
%     \item Zeno
%
%   \end{itemize}
% \end{frame}


\begin{frame}[fragile]
  \frametitle{Proof: \Large\texttt{rev (drop i xs) = take (len xs-i) (rev xs)}}
  \begin{center}
  \begin{tabular}{>{\small}c >{\small}l}
   No & Conjecture \\
   \hline
  1 & \verb!len (drop x xs) = len xs-x!               \\[-1pt]
  2 & \verb!len xs          = len (rev xs)!           \\[-1pt]
  3 & \verb!xs              = take x xs++drop x xs!   \\[-1pt]
  4 & \verb!rev (ys++xs)    = rev xs++rev ys!         \\[-1pt]
  5 & \verb!xs              = take (len xs) (xs++ys)!
  \end{tabular}
  \end{center}

  \begin{align*}
  & \hspace{-2em} \small \hs{rev (drop i xs)}                                                   & \small = \left\{ 5 \right\} \\[-3pt]
  & \hspace{-2em} \small \hs{take (len (rev (drop i xs))) (rev (drop i xs)++rev (take i xs))}   & \small = \left\{ 2 \right\} \\[-3pt]
  & \hspace{-2em} \small \hs{take (len (drop i xs)) (rev (drop i xs)++rev (take i xs))}         & \small = \left\{ 1 \right\} \\[-3pt]
  & \hspace{-2em} \small \hs{take (len xs-i) (rev (drop i xs)++rev (take i xs))}                & \small = \left\{ 4 \right\} \\[-3pt]
  & \hspace{-2em} \small \hs{take (len xs-i) (rev (take i xs++drop i xs))}                      & \small = \left\{ 3 \right\} \\[-3pt]
  & \hspace{-2em} \small \hs{take (len xs-i) (rev xs)}
  \end{align*}
\end{frame}

\begin{frame}[fragile]
  \frametitle{Future work: Conditional properties II}

  What about

  \begin{equation*}
    \hs{x < y} = \hs{True} \land
    \hs{y < z} = \hs{True} \implies
    \hs{x < z} = \hs{True}
  \end{equation*}

  \pause
  Same trick?

  \begin{verbatim}
    data Pair = Pair { smaller :: Nat , larger :: Nat }
  \end{verbatim}

  Can we state the property?

  \begin{equation*}
    \hs{smaller p1 < larger p2} = \hs{True}
  \end{equation*}

  \pause
  {\color{red} Problem:} how are \hs{p1} and \hs{p2} related?

\end{frame}

\begin{frame}[fragile]
  \frametitle{Limitation: Expensive calculations}

  Imagine a program which does exponentiation, \hs{**}, on
  unary nats \hs{data Nat = Zero | Succ Nat}

  \vspace{\baselineskip}

  Too expensive to caluclate \hs{x ** (y ** z)}.

  \vspace{\baselineskip}

  \pause

  \begin{verbatim}
  f x y = if y == 1000000000000
               then 0
               else x + y
  \end{verbatim}
\end{frame}

\end{comment}

\end{document}

