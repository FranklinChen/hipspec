\defverbatim{\cla}{%
\begin{verbatim}
xs
xs++[]
[]++xs
rot Z xs
rot (len xs) xs
\end{verbatim}
}

\defverbatim{\clb}{%
\begin{verbatim}
xs++ys
[]++(xs++ys)
rot Z (xs++ys)
(xs++ys)++[]
rot (len ys) (ys++xs)
\end{verbatim}
}

\defverbatim{\clc}{%
\begin{verbatim}
xs++(ys++ys)
(xs++ys)++ys
\end{verbatim}
}

\defverbatim{\cld}{%
\begin{verbatim}
rot n (xs++xs)
rot n xs++rot n xs
\end{verbatim}
}

\defverbatim{\cle}{%
\begin{verbatim}
len xs
len (rot n xs)
\end{verbatim}
}

\defverbatim{\clf}{%
\begin{verbatim}
len (xs++ys)
len (ys++xs)
\end{verbatim}
}

\defverbatim{\clg}{%
\begin{verbatim}
rot n (rot m xs)
rot m (rot n xs)
\end{verbatim}
}


\tikzstyle{eqblock} = [rectangle, draw=Purpleee, thick, text width=4.75em]
\newcommand\vb[1]{\mbox{{\tt #1}}}
\makebox[\textwidth][c]{%
\begin{tikzpicture}

\node at (0.5,2) [eqblock,text width=85] {\cla};

\node at (0.5,-2) [eqblock,text width=115] {\clb};

\node at (6,3) [eqblock,text width=65] {\clc} ;

\node at (6,1)  [eqblock,text width=100] {\cld};

\node at (8,-1)  [eqblock,text width=80] {\cle};

\node at (4.5,-1)  [eqblock,text width=70] {\clf};

\node at (6.5,-3)  [eqblock,text width=90] {\clg};

\end{tikzpicture}
}


